\documentclass[10pt]{scrartcl}
\usepackage[default]{Formatting}
\usepackage{amsmath}
\usepackage{calligra}
\usepackage[T1]{fontenc}
\usepackage{siunitx}
\allowdisplaybreaks
\title{LLNL Research}
\subtitle{Deuterium Normalization}
\author{Wentinn Liao}
\renewcommand\theauthor\thesubtitle
\date{Saturday, Jun 12th 2021}

\begin{document}
	\maketitle
	\slp{0}
	(d + f(E)g(V))c(E) + b(E)
	\[
		\log{\paren{P_E(\rho) / P_H(\rho)}} = c(E)(\log{\paren{P_D(\rho) / P_H(\rho)}}) + f(\rho, E)
	\]

	"Excess constant" in $\log{P_D / P_H}$ affected residual $f$ too much, so shift by $d$ such that $\log{\paren{P_D / P_H}} - d \rightarrow 0$
	
	\[
		\log{\paren{P_E / P_H}} = c(E)(\log{\paren{P_D / P_H}} - d) + f(\rho, E)
	\]
	
	Analysis on the residual function suggests that $\frac{f(\rho, E)}{c(E)}$ is inversely related to $m(E)$, i.e. $\frac{f(\rho, E)}{c(E)} = r + \frac{s(\rho)}{m(E)}$ for some constant $r$ and some function $s(\rho)$ independent of element
	
	\[
		\log{\paren{P_E / P_H}} = c(E)\paren{(\log{\paren{P_D / P_H}} - d + r) + \frac{s(\rho)}{m(E)}}
	\]
	
	Strategy: SVD and project onto principal component for crude approximation of $c(E)$. Divide by $c(E)$ and extract $\log{\paren{P_D / P_H}} + \mc{C}$ term, scale by $m(E)$ to see approximate shape of $s(\rho)$. Use $s(\rho)$ and least squares to get more accurate $c(E)$, and repeat process
\end{document}


















